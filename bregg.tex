\section{Брэгговская решётка}
После открытия способа получения решёток с показателем преломления в оптических волокнах
открыло совершенно новый класс светопроводящих устройств с маленькими вносимыми потерями и очень
гибким спектральным коэффицентом пропускания. Этому способствовало изобретение процесса внешней записи, где периодический профиль ультрафиолетового света полученный в следствии интерференции двух лучей используется для копирования периодической структуры интерференционной картины в областях волокн, где происходят фотохимические реакции между ультрафиолетовыми фотонами и стеклом.
\par
Эти так называемые Волоконные Брэгговские Решётки получившие своё название из-за сходства с голографической диффракционной решёткой Брегга в которой луч направлен в сердевину волокна.
Волоконные Брегговские Решётки широко используются для стабилизации лазерных диододов накачки, температурных сенсорах, акустике, при мультиплексировании длин волн, уплощения коэффицента усиления, компенсации рассеивания в оптических системах сообщения и всё больше в зеркалах резанаторов волоконных лазеров.

Волоконные брэгговские решётки, кроме всего прочего, представляют ещё и из-за того, что их коэффицент пропускания с высокой точностью очень легко настраиваем для нужд производства с наперёд заданными характеристиками.
В данной работе главную роль будет играть частный случай волоконных брегговских решёток а именно наклонные брэгговские решётки в литературе именуемые как  TFBG (Tilted Fiber Bragg Gratings).