\setcounter{equation}{0}

\section{Заключение}\label{slast}
\qquad В результате проделанной работы были уточненены уравнения движения комплексного скалярного поля в возмущенной метрике Фридмана-Робертсона-Уокера, полученные в работе L. Arturo Ureña-López \cite{SF}, найденные поправки будут существены при рассмотрении ненулевого порядка по $H/m$, так же были получены аналогичные уравнения для действительного поля. Из этих уравнений получены в предельном переходе уравнения Шредингера-Пуассона. Из анализа уравнений Шредингера-Пуассона получен Гамильтониан, дающий правильную оценку скорости релаксации комплексного и аксионного поля за счет гравитационного взаимодействия $\Gamma\sim\delta\rho$, последний результат опровергает результат полученный в работе Erken, O. and Sikivie, P. and Tam, H. and Yang, Q. \cite{Sikivie}.

\pagebreak[4]


\section*{Благодарности}\label{ty}

В первую очередь, хочется поблагодарить моего научного руководителя А.В. Дорофеенко без которого было бы невозможным написание данной работы и чья неоценимая помощь и руководство помогли её завершить. Автор также признателен А.П. Виноградову за лекции по электродинамике композоитов прочтённые им в ИТПЭ, которые помогли сформировать представления автора об этой новой области науки и чьи наставления значительно повлияли на мирощущение автора. Особенную благодарность заслуживает А.А. Пухов, сыгравший ключевую роль в образовании и становлении автора как учёного, за его лекции по квантовой механике, статистической физике, колебаниям и волнам живая и интересная подача которых возбудила неутомимый интерес автора к теоретической физике. Его вера в автора не давала сдаваться перед лицом трудностей, а благочестивые нравоучения направляли на пути познания. Кроме того, автор благодарен А.М. Мерзликину
за очень содержательные и познавательные семинары предмет которых затруднительно вспомнить. 


\pagebreak[4]