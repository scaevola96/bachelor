
\addcontentsline{toc}{section}{Введение}

\section*{Введение}\label{intro}

Трудно переоценить роль разработки новых экпериментальных приборов в современной физике. С каждым годом экспериментаторам и инженерам приходится изощряться для удоволетворения потребностей современной науки, техники и индустрии. Эти потребности включают не только обнажённый научный интерес учёных которые зачастую занимаются своей областью без оглядки на какие-либо приложения, но и совсем не эфемерные и утилитарные требования индустрии которая ставит всё новые планки для характеристик устройств нацеленных на коммерческое применение. В науке и технике есть целый спектр классов разных устройст каждое имеющее своё назначение. Нас будут интересовать, в частности, сенсоры (sensor от лат. корня sentire означающее "чувствовать"). В обиход это слов вошло с начала 50х годов  прошлого века
и пришло в русский язык из английского. Согласно Аристотелю, который заложил основы современного познания и науки, чтобы познать какой-то объект надо определить, так называемые, четыре причины Аристотеля. Не углубляясь в эпистемологический анализ сенсора рассмотрим одну из них, а именно, целевую причину. Цель или назначение сенсора состоит в том, чтобы на какое-то внешнее изменение физических величин дать соответвующий сигнал который позволит дать оценку мере этой самой физической величины. Во времена зарождения слова "сенсор" , приборы выполняющие данную функцию имели полупроводниковую или электрическую природу, но сегодня набирают популярность сенсоры которые основываются на оптоволоконных свойствах вещества. Сегодня, основным и наиболее коммерчески применимыми основываются на эффекте Кречмана и на явлении поверхностного плазмонного резонанса. В этой же работе мы сконцентрируемся на изучении иной конфигурации оптической системы, а именно наклонной волоконной решётки Брегга и её частного, но очень особенного и важного случая - наклонной волоконной решётки Брегга.


\section*{Благодарности}\label{ty}

В первую очередь, хочется поблагодарить моего научного руководителя А.В. Дорофеенко без которого было бы невозможным написание данной работы и чья неоценимая помощь и руководство помогли её завершить. Автор также признателен А.П. Виноградову за лекции по электродинамике композоитов прочтённые им в ИТПЭ, которые помогли сформировать представления автора об этой новой области науки и чьи наставления значительно повлияли на мирощущение автора. Особенную благодарность заслуживает А.А. Пухов, сыгравший ключевую роль в образовании и становлении автора как учёного, за его лекции по квантовой механике, статистической физике, колебаниям и волнам живая и интересная подача которых возбудила неутомимый интерес автора к теоретической физике. Его вера в автора не давала сдаваться перед лицом трудностей, а благочистивые нравоучения направляли на пути познания. Кроме того, автор благодарен А.М. Мерзликину
за очень содержательные и познавательные семинары предмет которых затруднительно вспомнить. 


\pagebreak[4]