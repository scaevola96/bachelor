\section{Эффект Кречмана}





Рассмотрим падение света на слоистую среду под углом $ 
\theta_1 $ с коэффицентами  диэлектрической проницаемостями $ \varepsilon_1, \varepsilon_2, \varepsilon_3 $ соответственно. 
На границе каждой среды можно записать закон Снеллиуса.

	$$
	\frac{\sin{\theta_2}}{\sin{\theta_1}} = \frac{n_2}{n_1} =\sqrt{\frac{\varepsilon_{2}}{\varepsilon_1}}	$$
	$$
	\frac{{\sin\theta_3}}{\sin{\theta_2}} = \frac{n_3}{n_2} =\sqrt{\frac{\varepsilon_{3}}{\varepsilon_2}}	
	$$

Наша цель посчитать коэффиценты отражения и прохождения для такой слоистой системы. В это задаче очень полезен метод $ T $-матриц. Обозначим коэффицент отражения через $ r $ , и коэффицент прохождения через $ t $.

Тогда на матричном языке можно задать взаимосвязь между этими величинами. А именно распишем ампплитуды волн слева и справа.

$$\vec{E}=\vec{E}(x) e^{i(\omega t-\beta z)}$$

$$E(x)=\left\{\begin{array}{c}
E_{1} e^{i k_{1} x}+E_{1}^{\prime} e^{-i k_{1} x}, x<0 \\
E_{2} e^{i k_{2 x} x}+E_{2}^{\prime} e^{-i k_{2} x_{2} x}, 0<x<d \\
E_{3} e^{i k_{3}(x-d)}+E_{3}^{\prime} e^{-i k_{3}(x-d)}, x>d
\end{array}\right.$$

$$P_{2}=\left(\begin{array}{cc}
e^{i k_{2 x} d} & 0 \\
0 & e^{-i k_{2 x} d}
\end{array}\right)$$



$$
S_{21} = 
\frac{1}{2 Z_2}\begin{pmatrix}
	Z_2+Z_1 & Z_2 -Z_1 \\
	Z_2-Z_1 & Z_2+Z_1
\end{pmatrix}
$$


$$
S_{32} = 
\frac{1}{2 Z_3}\begin{pmatrix}
Z_3+Z_2 & Z_3 -Z_2 \\
Z_3-Z_2 & Z_3+Z_2
\end{pmatrix}
$$

Где $Z_i = \frac{k_{zi}}{k_0}$ для $ s $ - поляризации и $Z_i = \frac{k_{zi}}{\varepsilon_i k_0}$ для $ p $ - поляризации 

$$
 k_{zi} = \sqrt{k_0^2 \epsilon_i - k_x^2)}
$$

$$
 k_{x} = k_0 \sin{ \theta}
$$


$$
S_{32} P_2 S_{21} \begin{pmatrix}
	1\\
	r 
\end{pmatrix} = 
\begin{pmatrix}
t\\
0 
\end{pmatrix}
$$

\begin{center}
	\begin{figure}[h]
		\centering
		\begin{tikzpicture}
		\begin{axis}[
		title={$SiO_2 + Ag + H_2 O$},
		line width=2,
		enlargelimits=false,
		ylabel=$ R $,
		xlabel=$k/k_0$,
		label style={font=\bfseries\Large},
		legend style={at={( 0.8,0.2)},
			anchor=north west, font=\small},
		grid=major]
		
		\addplot +[mark=none] coordinates {(1.331,0) (1.331, 1)};
		\addplot +[mark=none] coordinates {(1.5,0) (1.5, 1)};
		\addplot[smooth,] table {krechman650nm.dat};
		
		\end{axis}
		
		
		\end{tikzpicture}
		\caption{Эффект Кречмана для серебра $ \lambda = 650 \text{нм} $}
		\label{fig:krech1}
		
	\end{figure}
\end{center}


\begin{center}
	\begin{figure}[h]
		\centering
		\begin{tikzpicture}
		\begin{axis}[
		title={$SiO_2 + Ag + H_2 O$},
		line width=2,
		enlargelimits=false,
		ylabel=$ R $,
		xlabel=$k/k_0$,
		label style={font=\bfseries\Large},
		legend style={at={( 0.8,0.2)},
			anchor=north west, font=\small},
		grid=major]
		\addplot +[mark=none] coordinates {(1.317,0) (1.317, 1)};
		\addplot +[mark=none] coordinates {(1.5,0) (1.5, 1)};
		\addplot[smooth,] table {krechman1550nm.dat};
		
		\end{axis}
		
		
		\end{tikzpicture}
		\caption{Эффект Кречмана для серебра $ \lambda = 1550 \text{нм} $}
		\label{fig:krech2}
	\end{figure}
\end{center}

Решая эту систему уравнений относительно $ r $ получаем зависимость коэффицента отражения от угла падающей волны. Для $ p $ поляризации при определённой каллибровке толщины металлического слоя (условие на Кречманна???)
можно наблюдать, что при некотором угле амплитуда отражённой волны полностью зануляется, что часто называют нарушенным полным отражением или эффектом Кречманна.
Данный имеет коллосальный потенциал для приложений.






