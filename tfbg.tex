\section{Наклонная Волоконная Брэгговская Решётка}
Чтобы понять физику наклонных волокон удобно рассмотреть спектры коэффицентов 
пропускания и отражения пары идентичных решёток наклонённые относительно друг друга на $ 10 $ градусов.
Можно выделить два важных случая: cлучай нормальной волоконной брэгговской решётки у которой только один сильный резонанс. К примеру провал в спектре пропускания соответствует условию Брэгга для периода решётки в этом волокне и тот же резонанся появляется как одиничный пик в спектре отражения. Длина волна $ \lambda_{B} $ соответсвующая брэгговскому резонансу самая большая так эффективный показатель преломления для одной моды направленной вдоль сердцевины самый большой.

В дополнение к брэгговскому резонансу, спектр наклонной решётки Брэгга иммеет большое количество сопутствующих резонансов, но только в спектре пропускания. Эти резонансы возникают из-за взаимодействия мод друг с другом.Оболочечные моды не видны в спектре отражения так как исчезают из-за потерь в оболочке Когда решётка наклонена, самый существенный эффект это значительное усиление оболочечных мод за счёт основной.
Ближайший к брэгговскому резонанс обычно сильнее своего оболочечного соседа со коротковолновой стороны и называется "призрачной" модой так как по своим свойства она очень похожа на брегговский резонанс, но
является суперпозицией нескольких оболочечных мод малого порядка. В спектре пропускания наклонной решётки есть значение около $ 1530 $ нм где непрерывость в оболочечной моде теряется после которой (в коротко
волновую сторону) наблюдается резкое убываение резонансных амплитуд. Этот эффект объясняется переходом от направленных оболочечных мод к оболочечным модам с потерями.







\begin{figure}[h]
	\centering
	\includegraphics[width=0.7\linewidth]{screenshot001}
	\caption{}
	\label{fig:screenshot001}
\end{figure}


\par
Дифракция на решётке проходит эффективно тогда и только тогда удоволетворены два условия: закон сохранения импульса (иначе говоря фазы совпадают)


$$\beta_{i} \pm \beta_{G}=\beta_{j}$$ 

\begin{figure}
	\centering
	\includegraphics[width=0.7\linewidth]{screenshot005}
	\caption{}
	\label{fig:screenshot005}
\end{figure}


\begin{figure}[h]
	\centering
	\includegraphics[width=0.4\linewidth]{screenshot004}
	\caption{}
	\label{fig:screenshot004}
\end{figure}

\subsection{Собственные оптические свойства наклонной решётки}

С увеличением угла наклона огибающа взаимодействующих резонансов смещается в коротковолновую сторону. Самая часто используемая и полезная конфигурация удобная для химических сенсоров это угол в 10 градусов так как эффективный коэффицент преломления её преобладающей оболочечной резонансной моды перекрывает очень важную область около значения $ 1.3 $.
\begin{figure}
	\centering
	\includegraphics[width=0.7\linewidth]{screenshot006}
	\caption{}
	\label{fig:screenshot006}
\end{figure}


Очень сильные  резонансные взаимодействия (потеря пропускания около 20 $dB$ соответствующая захвату  $ 99\%   $ входного света оболочечной модой с определённой частотой) достигается с похожей шириной резонанса что и в нормальный решётке, примерно 0.1 нм для решётки длинной 1 см. Также возможно участить период решётки и тем самым расширить резонансы и даже сделать их полностью перекрывающимися получив гладкий пропускной фильтр. Абсолютные и относительные  амплитуды Брэгговского резонанса и призрачная мода значительно изменяются с углом наклона.

\par

Хотя брэгговсикй резонанс присутствует только в отражении в дальнейшем мы покажем, что отражающая оболочечная мода сенсора может быть имплементирована с помощб разных связующих сред для области сердцевина-оболочка. Другой опцией является использование наклонной решётки в отражающей конфигурации поставив рефлектор далельше от решётки так чтобы свет прошёл бы через наклонную решётку дважду и вернулся бы к источнику. В своём простейшем исполнении, хорошая щель обеспечит широкополостное отражение пары процентов падающего света достаточное для измерения пропускных резонансов оболочечной моды. Для более эффективного использования света, можно накрыть зеркальной прослойкой (золото или серебро) на последующей выемке для 100\% отражения падающего света. Другой вариант для отражателя может быть сделан и обычной прямой брэгговской решётки с подобранным спектром отражения так чтобы желанный волновой диапазон спектра наклонной решётки отражался как показано на рисунке.

\begin{figure}[h]
	\centering
	\includegraphics[width=0.7\linewidth]{screenshot007}
	\caption{}
	\label{fig:screenshot007}
\end{figure}

\par 
Ещё одным важным параметром является поляризация света. Поляризация очень сильно влияет на спетра наклонной решётки










\section{Теория Наклонной Брегговской решётки}
\subsection{Влияние угла наклона, длины и силы решётки}
 Для начала можно  переписать условие совпадения фаз в более удобной форме.
 Так для длины волны $ \lambda_{r} $ резонанся решётки между основной моды и другой обозначенной через $ r $
$$\lambda_{r}=\left(N_{\mathrm{eff}}^{\mathrm{core}}\left(\lambda_{r}\right)+N_{\mathrm{eff}}^{r}\left(\lambda_{r}\right)\right) \Lambda / \cos (\theta)$$

 Где $ \Lambda $ период интерференционных полос используемая для создания решётки, $ \theta $ угол наклона плоскости решётки относительно плоскости поперечного сечения.
 $N_{\mathrm{eff}}^{\mathrm{core}}\left(\lambda_{r}\right)$ эффективный коэффицент преломления одной моды направленной через сердцевину на длине волны на которой наблюдается резонанся $ \lambda_{r} $ b $$N_{\mathrm{eff}}^{\mathrm{r}}\left(\lambda_{r}\right)$$ эффектиный коэффицент преломления моды $ r $ на той же длине волне. Важно учитывать дисперсию так как резонансы оболочечной моды возникают на сравнительно широком спектральном промежутке (Более 100 нм в случае угла наклона в 10 градусов).
 
\par
Cила решёточного резонанса (отражательная мощность $ R $) зависит от коэффицента перекрытия $ \kappa $ между 
падающей основной модой и той модой которая совпадает с ней по фазе.
Так для решётки длинной $ L $ Отражательная мощность выражается через




$$R=\tanh ^{2}(\kappa L)$$


$$\kappa=C \iint_{-\infty}^{\infty} \vec{E}_{\mathrm{core}}^{*} \cdot \vec{E}_{r} \Delta \mathrm{n}(\mathrm{x}, \mathrm{y}) \mathrm{d} \mathrm{xdy}$$


где $ С $ коэффицент пропорициональности связанный с нормировкой $ E_{core} $ и $ E_{r}$ и  $\Delta \mathrm{n}(\mathrm{x}, \mathrm{y})$ - функция описывающая возмущение коэффицента преломления из-зп присутствия решётки в поперечном сечении волокна
 
 \begin{figure}[h]
 	\centering
 	\includegraphics[width=0.7\linewidth]{screenshot008}
 	\caption{}
 	\label{fig:screenshot008}
 \end{figure}
 