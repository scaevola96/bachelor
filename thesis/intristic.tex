
\section{Cобственные свойства сенсора}

Для наклонной решётки, длины волн брэгговской и $ r $ - ой оболочечной моды
вследствие аксиального сжатия $ \Delta \varepsilon $ и изменения температуры $ \Delta T $ и могут быть выведены из следующих уравнений.



$$
\begin{aligned}
\Delta \lambda_{\mathrm{B}}=\left(2 \frac{N_{\mathrm{eff}}^{\mathrm{core}}}{\cos (\theta)} \frac{d \Lambda}{d \varepsilon}+2 \frac{\Lambda}{\cos (\theta)} \frac{d N_{\mathrm{eff}}^{\mathrm{core}}}{d \varepsilon}\right) \Delta \varepsilon \\
+\left(2 \frac{N_{\mathrm{eff}}^{\mathrm{core}}}{\cos (\theta)} \frac{d \Lambda}{d T}+2 \frac{\Lambda}{\cos (\theta)} \frac{d N_{\mathrm{eff}}^{\mathrm{core}}}{d T}\right) \Delta T 
\end{aligned}
$$



$$
\begin{aligned}
\Delta
 \lambda^{\mathrm{r}}=\left(\frac{\left(N_{\mathrm{eff}}^{\mathrm{core}}+N_{\mathrm{eff}}^{r}\right)}{\cos (\theta)} \frac{d \Lambda}{d \varepsilon}+\frac{\Lambda}{\cos (\theta)} \frac{d\left(N_{\mathrm{eff}}^{\mathrm{core}}+N_{\mathrm{eff}}^{r}\right)}{d \varepsilon}\right) \Delta \varepsilon  \\
+\left(\frac{\left(N_{\mathrm{eff}}^{\mathrm{core}}+N_{\mathrm{eff}}^{r}\right)}{\cos (\theta)} \frac{d \Lambda}{d T}\right.
\left.+\frac{\Lambda}{\cos (\theta)} \frac{d\left(N_{\mathrm{eff}}^{\mathrm{core}}+N_{\mathrm{eff}}^{r}\right)}{d T}\right) \Delta T
\end{aligned}
$$

Для температурной зависимости доминирует термо-оптическое слагаемое которое мало из-за сравнительно маленького температурного коэффицента расширения кремния $\left(0.5 \times 10^{-6} /^{\circ} \mathrm{C}\right)$


Тогда разность между сдвигами длин волн $\Delta \lambda_{\mathrm{B}}$  и $\Delta \lambda^{r}$

cтановится пропорцианальной очень маленькому фактору. $\frac{d\left(N_{\mathrm{eff}}^{\mathrm{core}}-N_{\mathrm{eff}}^{r}\right)}{ d T}$.

Это очень важный результат означающий что спектр наклонной решётки Брэгга инвариантен относительно температурных изменений


Для аксиального сжатия
\begin{equation}\Delta \lambda_{\mathrm{B}}-\Delta \lambda^{r}=\left(\frac{\left(N_{\mathrm{eff}}^{\mathrm{core}}-N_{\mathrm{eff}}^{r}\right)}{\cos (\theta)} \frac{d \Lambda}{d \varepsilon}\right) \Delta \varepsilon\end{equation}


В результате относительный сдвиг получается возрастающе большим для мод высокого порядка у которых коэффицент преломления значительно меньше чем у сердцевины и резонансы оболочечных мод отстают относительно брегговских когда волокно растягивается.
