\section{Моды волновода}

Для расчёта мод в волноводе имеющую форму цилиндра и состоящую из двух слоёв с показателями преломления $ n_1 $ и $ n_2 $ удобно перейти к цилиндрической систем координат. Так-как предполагается что волновод  однороденны вдоль оси цинлиндра зависисмотсь от $ z $ и $ t $ может быть взята в виде $\exp (i h z-i \omega t)$
Запишем Уравнения Максвелла:
$$\operatorname{rot} \mathbf{H}=\frac{\partial D}{\partial t}, \quad \operatorname{rot} \mathbf{E}=-\frac{\partial B}{\partial t}$$

В цилиндрической системе координат они примут вид:

$$\frac{1}{r} \frac{\partial H_{z}}{\partial \varphi}-\frac{\partial H_{\varphi}}{\partial z}=-i \omega \varepsilon E_{r}, \frac{\partial H_{r}}{\partial z}-\frac{\partial H_{z}}{\partial r}=-i \omega \varepsilon E_{\varphi}$$

$$\frac{1}{r} \frac{\partial}{\partial r}\left(r H_{\varphi}\right)-\frac{1}{r} \frac{\partial H_{r}}{\partial \varphi}=-i \omega e E_{z}$$

$$\frac{1}{r} \frac{\partial E_{z}}{\partial \varphi}-\frac{\partial E_{\varphi}}{\partial z}=i \omega \mu H_{r}, \quad \frac{\partial E_{r}}{\partial z}-\frac{\partial E_{z}}{\partial r}=i \omega \mu H_{\varphi}$$

$$\frac{1}{r} \frac{\partial E_{z}}{\partial \varphi}-\frac{\partial E_{\varphi}}{\partial z}=i \omega \mu H_{r}, \quad \frac{\partial E_{r}}{\partial z}-\frac{\partial E_{z}}{\partial r}=i \omega \mu H_{\varphi}$$

$$\frac{1}{r} \frac{\partial}{\partial r}\left(r E_{\Phi}\right)-\frac{1}{r} \frac{\partial E_{r}}{\partial \varphi}=i \omega \mu H_{z}$$


Из этих уравнений поперечные компоненты полей $E_{r}, E_{\varphi}, H_{r}, H_{\Phi}$

можно записать через продольные составляющие $E_{z}, H_{z}$

$$E_{r}=\frac{1}{k^{2}-h^{2}}\left(ih \frac{\partial E_{z}}{\partial r}+\frac{i \omega \mu}{r} \frac{\partial H_{s}}{\partial \varphi}\right)$$

$$E_{\varphi}=\frac{1}{k^{2}-h^{2}}\left(\frac{i h}{r} \frac{\partial E_{z}}{\partial \varphi}-i \omega \mu \frac{\partial H_{z}}{\partial r}\right)$$

$$H_{r}=\frac{1}{k^{2}-h^{2}}\left(\text { ih } \frac{\partial H_{\mathbf{z}}}{\partial r}-\frac{i \omega \mathbf{e}}{r} \frac{\partial E_{\mathbf{z}}}{\partial \varphi}\right)$$


$$H_{\varphi}=\frac{1}{k^{2}-h^{2}}\left(\frac{i h}{r} \frac{\partial H_{z}}{\partial \varphi}+i \omega e \frac{\partial E_{z}}{\partial r}\right)$$


Продольные компоненты полей удоволетворяют волновому уравнению

$$\frac{\partial^{2} U}{\partial r^{2}}+\frac{1}{r} \frac{\partial U}{\partial r}+\frac{1}{r^{2}} \frac{\partial^{2} U}{\partial \varphi^{2}}+\left(k^{2}-h^{2}\right) U=0$$


$$U=F(r) e^{t m \varphi}$$


$$u^{2}=k^{2}-h^{2}$$

$$\frac{\partial^{2} F}{\partial r^{2}}+\frac{1}{r} \frac{\partial F}{\partial r}+\left(u^{2}-\frac{m^{2}}{r^{2}}\right) F=0$$


$$E_{z}^{(t)}=\sum_{m=-\infty}^{\infty} A_{m} J_{m}(u r) \cos m \varphi \exp (i h z-i \omega t)$$


$$H_{z}^{(1)}=\sum_{m=-\infty}^{\infty} B_{m} J_{m}(u r) \cos \left(m \varphi+\beta_{m}\right) \exp (i h z-i \omega t)$$

$$E_{z}^{(2)}=\sum_{m=-\infty}^{\infty} C_{m} K_{m}(v r) \cos m \varphi \exp (i h z-i \omega t)$$

$$H_{z}^{(2)}=\sum_{m=-\infty}^{\infty} D_{m} K_{m}(v r) \cos \left(m \varphi+\beta_{m}\right) \exp (i h z-i \omega t)$$


$$
E_{\varphi}^{(1)} =-\sum_{m=-\infty}^{\infty}\left[A_{m} \frac{i m h}{u^{2} r} J_{m}(u r) \sin m \varphi-\right.
\left.-B_{m} \frac{i \omega \mu}{u} J_{m}^{\prime}(u r) \cos \left(m \varphi+\beta_{m}\right)\right]
$$

$$
H_{\varphi}^{(1)}=- \sum_{m=-\infty}^{\infty}\left[B_{m} \frac{i m h}{u^{2} r} J_{m}(u r) \sin \left(m \varphi+\beta_{m}\right)+A_{m} \frac{i \omega e_{1}}{u} J_{m}(u r) \cos m \varphi\right]
$$


$$
E_{\varphi}^{(2)}=\sum_{m=-\infty}^{\infty}\left[C_{m} \frac{i m h}{v^{2} r} K_{m}(v r) \sin \left(m \varphi+\beta_{m}\right)-D_{m} \frac{i \omega \varepsilon_{2}}{v} K_{m}^{\prime}(v r) \cos m \varphi\right]
$$

$$
H_{\varphi}^{(2)}= \sum_{m=-\infty}^{\infty}\left[D_{m} \frac{i m h}{v^{2} r} K_{m}\left(v^{2} r\right) \sin \left(m \varphi+\beta_{m}\right)-C_{m} \frac{i \omega \varepsilon_{2}}{v} K_{m}^{\prime}(v r) \cos m \varphi\right]
$$


На границе $ r = a $ тангенциальные компоненты векторов 
напряженности поля непрерывны. Из этого требования  
получаем 


$$-A_{m} \frac{m h}{p^{2}} J_{m}(p) \sin m \varphi-B_{m} \frac{\omega \mu}{p} J_{m}^{\prime}(p) \cos \left(m \varphi+\beta_{m}\right)=$$

$$=C_{m} \frac{m h}{q^{2}} K_{m}(q) \sin m \varphi+D_{m} \frac{\omega \mu}{q} K_{m}^{\prime}(q) \cos \left(m \varphi+\beta_{m}\right)$$

$$A_{m} J_{m}(p)=C_{m} K_{m}(q)$$

$$-B_{m} \frac{m h}{p^{2}} J_{m}(p) \sin \left(m \varphi+\beta_{m}\right)+A_{m} \frac{\omega \varepsilon_{1}}{p} J_{m}^{\prime}(p) \cos m \varphi=$$

$$=D_{m} \frac{m h}{q^{2}} K_{m}\left(q_{1}\right) \sin \left(m \varphi+\beta_{m}\right)-C_{m} \frac{\omega \varepsilon_{2}}{q} K_{m}^{\prime}(q) \cos m \varphi$$

$$B_{m} J_{m}(p)=D_{m} K_{m}(q)$$

$$\frac{\left[f_{m}(p)+g_{m}(q)\right]\left[\frac{\varepsilon_{1}}{\varepsilon_{0}} f_{m}(p)+g_{m}(q)\right]}{\frac{m^{2} h^{2}}{k_{2}^{2}}\left(\frac{1}{p^{2}}+\frac{1}{q^{2}}\right)}
=\frac{\sin m \varphi \sin \left(m \varphi+\beta_{m}\right)}{\cos m \varphi \cos \left(m \varphi+\beta_{m}\right)}$$


где

$$f_{m}(p)=\frac{J_{m}^{\prime}(p)}{p J_{m}(p)}, \quad g_{m}(q)=\frac{K_{m}^{\prime}(q)}{q K_{m}(q)}$$

$$p=u a, q=v a$$

$$p^{3}+q^{2}=a^{2}\left(k_{1}^{2}-k_{2}^{2}\right)$$

$$\left[f_{m}(p)+g_{m}(q)\right]\left[\frac{\varepsilon_{1}}{\varepsilon_{2}} f_{m}(p)+g_{m}(q)\right]=\frac{m^{2} h^{2}}{k_{2}^{2}}\left(\frac{1}{p^{2}}+\frac{1}{q^{2}}\right)^{2}$$